% !TEX root = report.tex

\chapter{\label{ch:summary}Report High-Level Summary}
I was tasked with performing an internal penetration test towards Offensive
Security Exam. An internal penetration test is a dedicated attack against
internally connected systems. The focus of this test is to perform attacks,
similar to those of a hacker and attempt to infiltrate Offensive Security's
internal exam systems - the THINC.local domain. My overall objective was to
evaluate the network, identify systems, and exploit flaws while reporting
the findings back to Offensive Security.

When performing the internal penetration test, there were several alarming
vulnerabilities that were identified on Offensive Security's network. When
performing the attacks, I was able to gain access to multiple machines,
primarily due to outdated patches and poor security configurations. During
the testing, I had administrative level access to multiple systems. All
systems were successfully exploited and access granted. These systems as
well as a brief description on how access was obtained are listed below:

\begin{itemize}
  \item 192.168.xx.xx (hostname) - Name of initial exploit
  \item 192.168.xx.xx (hostname) - Name of initial exploit
  \item 192.168.xx.xx (hostname) - Name of initial exploit
  \item 192.168.xx.xx (hostname) - Name of initial exploit
  \item 192.168.xx.xx (hostname) - BOF
\end{itemize}

\section{\label{sec:recommendations}Recommendations}
I recommend patching the vulnerabilities identified during the testing to
ensure that an attacker cannot exploit these systems in the future. One thing
to remember is that these systems require frequent patching and once patched,
should remain on a regular patch program to protect additional vulnerabilities
that are discovered at a later date.
